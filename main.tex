\documentclass[spanish]{article}
\usepackage[T1]{fontenc}
\usepackage[utf8]{inputenc}
\usepackage[a4paper]{geometry}
\geometry{verbose,tmargin=2cm,bmargin=2cm,lmargin=2cm,rmargin=2cm}
\setlength{\parindent}{0\paperheight}
\usepackage{color}
%\usepackage{xcolor}
\usepackage[dvipsnames]{xcolor}
\usepackage{amsmath}
\usepackage{listings}
\usepackage{graphicx}
\usepackage{esint}
\usepackage{verbatim}
\usepackage{colortbl}
\usepackage{babel}
\usepackage{enumitem}
\usepackage{multicol}
\usepackage{amssymb, amsmath}
\usepackage{tikz}
\usepackage{physics}
\usepackage{booktabs}
\usepackage{subcaption}
\usetikzlibrary{arrows,decorations.markings,plotmarks}
\selectlanguage{spanish}
\usepackage{caption}
\usepackage{circuitikz}
\usepackage{multicol}
\usepackage{multirow}
\usepackage{svg}
\usepackage{fancybox}
\usepackage{enumitem}
\definecolor{verdeodi}{RGB}{53,113,105} %%%%%%%
\usepackage[font={color=verdeodi,bf},figurename=Fig.,labelfont={it}]{caption}
\usepackage[most]{tcolorbox}
\usetikzlibrary{spy}
\newtcolorbox{fancybox}[1][]{ %%%%%%%
enhanced,
boxrule=1pt,arc=8pt,boxsep=0pt,
left=0.8em,right=0.8em,top=1ex,bottom=1ex,colback=verdeodi!20,colframe=verdeodi,#1}



\title{Subir archivos a GitHub por terminal}

\begin{document}
\maketitle
% Mostrar todo el texto (wrap): Alt + z / option + z


\section{Crear repositorio en GitHub web}

Se crea el repositorio nombrandolo igual que la carpeta con los archivos.

\textcolor{red}{NO CREAR UN} \textcolor{orange}{README.md} \textcolor{orange}{.gitignore} 

\section{Copiar el enlace del repositorio}

\section{Crear un README.md y .gitignore}

Para ello en la terminal usamos

\begin{center}
    \Large\textcolor{blue}{touch README.md} o \Large\textcolor{blue}{touch .gitignore} 
\end{center}

Modificamos los archivos usando

\begin{center}
    \Large\textcolor{blue}{open .gitignore} o \Large\textcolor{blue}{open README.md}  
\end{center}

\section{Comprobar el estado de git dentro del terminal de la carpeta}

\begin{center}
    \Large{\textcolor{blue}{git status}}    
\end{center}

\vspace{3mm}

\textcolor{black!50!red!50!}{\textbf{\large Si no hay nada}}

\begin{center}
    \Large{\textcolor{blue}{git init}}
    
    \Large{\textcolor{blue}{git add} \textcolor{green!20!black!80!}{[archivos a copiar]}}

    \small(colocando \textcolor{white!20!blue!80!}{git add .} selecciona todo)

    \small (si se quiere una carpeta poner \textcolor{green!20!black!80!}{nombre-carpeta/})

    \Large{\textcolor{blue}{git remote add origin} \textcolor{green!20!black!80!}{https://github.com/usuario/repositorio.git}}
\end{center}

\section{Crear un commit con una breve descripcion}

\begin{center}
    \Large{\textcolor{blue}{ git commit -m `descripcion'}}
\end{center}

\section{Subir archivos}

\begin{center}
    \Large{\textcolor{blue}{ git push -u origin master}}
\end{center}

\section{Eliminar algo del repositorio}


\begin{lstlisting}{language=bash}
    find . -name ARCHIVO -print0 | xargs -0 git rm -f --ignore-unmatch
\end{lstlisting}

 

\end{document}